\documentclass[12pt]{article}
\usepackage[utf8]{inputenc}
\usepackage[russian]{babel} %comment it for english!
\usepackage{amsfonts,longtable,amssymb,amsmath,array}
\usepackage{graphicx}
\usepackage{euscript}
\usepackage{graphicx}
\graphicspath{ {images/} }
\newtheorem{vkTheorem}{Theorem}[section]
\newtheorem{vkLemma}[vkTheorem]{Lemma}
\newenvironment{vkProof}%
   {\par\noindent{\bf Proof.\par\nopagebreak}}%
   {\hfill$\scriptstyle\blacksquare$\par\medskip}
%\textwidth=450pt%%
%\textheight=650pt
%\oddsidemargin=0pt
%\hoffset=0pt
%\voffset=0pt
%\topmargin=0pt
%\headheight=0pt
%\headsep=0pt
\newcommand{\suml}[0]{\sum\limits}
\begin{document}


\section*{Арзуманян Виталий, CS, 2 курс}

\subsection*{Задача 4}
График зависимости средней абсолютной ошибки от количества нейронов скрытого слоя. Оптимальное значение достигается при $p=14$.

\includegraphics[scale=0.5]{1.png}

График зависимости ошибки от количества сетей в случае добавления в произвольном порядке:

\includegraphics[scale=0.7]{2.png}

График зависимости ошибки от количества сетей в случае добавления в порядке возрастания ошибки на обучающем множестве:

\includegraphics[scale=0.7]{3.png}

Видно, что в случае последовательного добавления ошибка падает ниже и дольше находится на низком уровне.

\subsection*{Задача 5}
\subsubsection*{a)}
$$\widehat{f}(x) = \frac{1}{n}\sum\frac{1}{h}K\Bigl(\frac{x - X_i}{h}\Bigr)$$
Проделаем выкладки:
$$\mathbb{E}\widehat{f}(x) = \frac{1}{n} \sum \int \frac{1}{h} K\Bigl(\frac{x-t}{h}\Bigr)f(t)dt = \int \frac{1}{h} K \Bigl( \frac{x - t}{h} \Bigr) f(t)dt = \frac{1}{h}\int\limits_{x - \frac{h}{2}}^{x + \frac{h}{2}}f(t)dt$$
$$\mathbb{V}\widehat{f}(x) = \frac{1}{n} \mathbb{V}\Bigl(\frac{1}{h}K\Bigl(\frac{x - X}{h}\Bigl)\Bigl) = \frac{1}{n} \Bigl( \int\frac{1}{h^2} K^2\Bigl(\frac{x - t}{t} \Bigr)f(t)dt - \mathbb{E}^2 \Bigr) =$$
$$= \frac{1}{nh^2} \Bigl( \int\limits_{x - \frac{h}{2}}^{x + \frac{h}{2}} f(t)dt - \bigl( \int\limits_{x - \frac{h}{2}}^{x + \frac{h}{2}} f(t)dt \bigr)^2 \Bigr)$$

\subsubsection*{b)}
$$\widehat{f}(x) = \frac{1}{n}\sum\frac{1}{h}K\Bigl(\frac{x - X_i}{h}\Bigr)$$
$h\to0, nh\to\infty$ при $n\to\infty$




\subsection*{Задача 6}
Из набора [0.0000025, 0.000005, 0.00001, 0.0001, 0.0002, 0.0005, 0.001, 0.005, 0.01, 0.02, 0.05, 0.1, 0.2, 0.25, 0.5, 1, 2, 5, 10] минимальные значения оценки риска и для гистограммы и для ядерной оценки достигаются на значении $h=0.0000025$.


\begin{flushright}
$\Box$
\end{flushright}








\end{document} 
